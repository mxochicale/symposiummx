\documentclass[a4paper]{article}


\usepackage{fancyhdr}
\pagestyle{fancy}
\fancyhf{}
\rhead{}
\lhead{XV Symposium of Mexican Studies and Students in the United Kingdom}
\renewcommand{\headrulewidth}{0pt}



%% Language and font encodings
\usepackage[english]{babel}
\usepackage[utf8x]{inputenc}

%% Sets page size and margins
\usepackage[a4paper,top=3cm,bottom=2cm,left=3cm,right=3cm,marginparwidth=1.75cm]{geometry}


\usepackage{lineno}
\linenumbers


%% Useful packages
\usepackage{amsmath}
\usepackage{graphicx}
\usepackage[colorinlistoftodos]{todonotes}
\usepackage[colorlinks=true, allcolors=blue]{hyperref}

% \usepackage{mathdots} % to use \iddots for an inverse-diagonal dots

% \title{Can a Robot tell you how well you move?}
% \title{How well you move?: Can a Robot asnwer that?}
%\title{How well you move? Movement Variability in the context of Human-Robot Interaction}
%\title{Towards the use of Humanoid Robots to analyse Human Movement}
% \title{Towards the analysis of Human Movement with the help of Humanoid Robots}
\title{Towards the Analysis of Human Movement with \\ Humanoid Robots}

% to analyse Human Movement}
\author{Miguel P. Xochicale, Chris Baber and Mourad Oussalah}
\date{Deadline: 28th of April 2017\\ \today }

\begin{document}
\maketitle
\thispagestyle{fancy}


\section*{Summary of research [1500 words]}

% %%%%%%%%%%%%%%%%%%%%%%%%%%%%%%%%%%%%%%%%%%%%%%%%%%%%%%%%%%%%%%%%%%%%%%%%%%%%%%%
% %%%%%%%%%%%%%%%%%%%%%%%%%   First Version
%
% The World Health Organization (WHO) pointed out that people worldwide are living longer.
% In 2015, 125 million people worldwide were aged 80 years or older
% and by 2050 there will be almost 434 million people in this age group worldwide
% of which 80\% of all older people will live in low- and middle-income countries.
% Similarly, WHO highlighted that the improvement of methodologies for measurement, monitoring
% and understanding the elderly is a priority area of action.
% With this, it is believed that elderly care using robot is possible.
% For instance, (a) RI-MAN humanoid robot can carry patients and transport them,
% also has facial recognition and scent discerner; (b) RIBA-II humanoid robot can carry up to 80 kg;
% (b) Paro bot helps people with dementia to decrease stress or feelings of loneliness; and
% (d) Palro humanoid robot can be used in entertainment activities such as dancing or gaming.
% Recently, humanoid robots like Pepper and NAO have been used to understand human emotions
% or to perform therapies for rehabilitation with children or elderly people.
% For this talk, I will therefore present the  methodologies for measurement, monitoring and
% understanding Human-Robot Interaction with the use of wearable inertial sensors.
% I will also present some application scenarios where a group of people is interacting
% with NAO and show the remarkable capabilities of measuring human body movement as
% well as the peer-to-peer influence when receiving instructions from the NAO.


% %%%%%%%%%%%%%%%%%%%%%%%%%%%%%%%%%%%%%%%%%%%%%%%%%%%%%%%%%%%%%%%%%%%%%%%%%%%%%%%
% %%%%%%%%%%%%%%%%%%%%%%%%%   Second Version
% The World Health Organization (WHO) pointed out that people worldwide are living longer.
% In 2015, 125 million people worldwide were aged 80 years or older
% and by 2050 there will be almost 434 million people in this age group worldwide,
%  of which 80\% will live in low- and middle-income countries.
% Similarly, WHO highlighted that the improvement of methodologies for measurement, monitoring
% and understanding the elderly is a priority area of action.
% With this in mind, we believe that we can address those areas of opportunity
% where humanoids robots can be used for elderly care.
% For instance, (a) RI-MAN humanoid robot can carry patients and transport them,
% also has facial recognition and a scent discerner; (b) RIBA-II humanoid robot can carry people up to 80 kg;
% (b) Paro bot helps people with dementia to decrease stress or feelings of loneliness; and
% (d) Palro humanoid robot can be used in entertainment activities such as dancing or gaming.
% Recently, humanoid robots like Pepper and NAO have been used to understand human emotions
% or to perform therapies for rehabilitation with children or elderly people.
% In this talk, I will therefore present the  methodologies for measurement, monitoring and
% understanding the Human-Robot Interaction with the use of wearable inertial sensors.
% I will also present some results of a group of persons interacting with NAO to show
% the remarkable capabilities of measuring peer-to-peer influence when receiving
% instructions from the NAO.


% %%%%%%%%%%%%%%%%%%%%%%%%%%%%%%%%%%%%%%%%%%%%%%%%%%%%%%%%%%%%%%%%%%%%%%%%%%%%%%%
% %%%%%%%%%%%%%%%%%%%%%%%%%   Third Version
% The World Health Organization (WHO) pointed out that people worldwide are living longer.
% In 2015, 125 million people worldwide were aged 80 years or older
% and by 2050 there will be almost 434 million people in this age group worldwide,
%  of which 80\% will live in low- and middle-income countries.
% Similarly, WHO highlighted that the improvement of methodologies for measurement,
% monitoring and understanding the elderly are a priority area of action.
% With this in mind, we believe that we can address those areas of opportunity
% where humanoids robots can be used for elderly care.
% For instance, (a) RI-MAN humanoid robot can carry patients and transport them,
% also has facial recognition and a scent discerner;
% (b) RIBA-II humanoid robot can carry people up to 80 kg;
% (c) Paro bot helps people with dementia to decrease stress or feelings of loneliness; and
% (d) Palro humanoid robot can be used in entertainment activities such as dancing or gaming.
% Recently, humanoid robots like Pepper and NAO have been used to understand
% human emotions or to perform therapies for rehabilitation with children or
% elderly people.
% In this talk, I will therefore present the  methodologies for measurement,
% monitoring and understanding the Human-Robot Interaction with the use of wearable
% inertial sensors.
% I will also present some results of a group of persons interacting with NAO
% to show the remarkable capabilities of measuring peer-to-peer influence when
% receiving instructions from the NAO.

%%%%%%%%%%%%%%%%%%%%%%%%%%%%%%%%%%%%%%%%%%%%%%%%%%%%%%%%%%%%%%%%%%%%%%%%%%%%%%%
%%%%%%%%%%%%%%%%%%%%%%%%%   Fourth Version
% (Amendments of Mourad Oussalah) 6 April 2017, 07:14
The World Health Organization (WHO) pointed out that people worldwide are living longer.
In 2015, 125 million people worldwide were aged 80 years or older
and by 2050 there will be almost 434 million people in this age group worldwide,
 of which 80 \% will live in low- and middle-income countries.
Similarly, WHO highlighted that the improvement of methodologies for measurement,
monitoring and understanding the elderly are a priority area of action.
With this in mind, we believe that we can address those areas of opportunity
where humanoids robots can be used for elderly care.
For instance, (a) RI-MAN humanoid robot
has facial recognition and a scent discerner with the ability to
% can
carry patients to different sorrounding locations;
% and transport them,
% also has facial recognition and a scent discerner;
(b) RIBA-II humanoid robot can carry people up to 80 kg;
(c) Paro bot helps people with dementia to decrease stress or feelings of loneliness; and
(d) Palro humanoid robot can be used in entertainment activities such as dancing or gaming.
Recently, humanoid robots like Pepper and NAO have been used to understand
human emotions, or to perform therapies for rehabilitation with children or
elderly people.
In this talk, I will therefore present the  methodologies for measurement,
monitoring and understanding the Human-Robot Interaction with the use of wearable
inertial sensors.
I will also present some results of a group of persons interacting with NAO
to show the remarkable capabilities of measuring peer-to-peer influence when
receiving instructions from the NAO.


Guneysu et al. 2015 \cite{guneysu2015children}.


%%%%%%%%%%%%%%%%%%%%%%%%%
%%%%% REFERENCES
\bibliographystyle{alpha}
\bibliography{references}

\end{document}
